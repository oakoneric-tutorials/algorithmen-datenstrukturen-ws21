\documentclass{beamer}
\usepackage{../tut-slides}
\usepackage{../mathoperatorsAuD}

\usepackage{lmodern}
\usepackage{amsmath,amssymb}
\usepackage{wasysym}
\usepackage{stmaryrd}
\usepackage{enumerate}
%\usepackage[inline]{enumitem} 		%customize label
%\newcommand{\labelitemi}{\raisebox{1pt}{\scalebox{.9}{$\blacktriangleright$}}}
%\newcommand{\labelitemii}{$\vartriangleright$}
%\newcommand{\labelitemiii}{--}
\setbeamertemplate{itemize item}{\raisebox{1pt}{\scalebox{.9}{$\blacktriangleright$}}}
\setbeamertemplate{itemize subitem}{$\vartriangleright$}

\usepackage{booktabs}
\usepackage{tabularx}
\usepackage{tabu}
\newcommand*\head{\rowfont{\bfseries}}
\newcommand*{\tw}{\rowfont{\ttfamily}}
\renewcommand{\tabularxcolumn}[1]{>{\hspace{0pt}}m{#1}}
\usepackage{multirow}

\usepackage{cancel}

\usepackage{empheq}
\newcommand*\widefbox[1]{\fbox{\hspace{2em} #1 \hspace{2em}}}

\usepackage{tcolorbox}
\newtcolorbox{mymathbox}[1][]{colback=white, sharp corners, #1}

\usepackage{xcolor}
\usepackage{listings}
\usepackage{bold-extra}
\newcommand*{\ttfamilywithbold}{\fontfamily{lmtt}\selectfont}

\lstset{numbers=left, 
	numberstyle=\tiny, 
	breaklines=true,
	%	backgroundcolor=false,
	numbersep=5pt,
	language=C,
	tabsize=2,
	basicstyle=\footnotesize\ttfamilywithbold,
	showstringspaces=false,
	frame=tb} 

\lstdefinestyle{notebook}{
	basicstyle=\footnotesize\ttfamilywithbold,   
	breakatwhitespace=false,         
	breaklines=true,                 
	commentstyle=\itshape, 
	escapeinside={\%*}{*)},          % if you want to add LaTeX within your code
	extendedchars=true,              % lets you use non-ASCII characters; for 8-bits encodings only, does not work with UTF-8
	backgroundcolor=\color{cdgray!10},
	frame=single,
	keywordstyle=\bfseries,       % keyword style
	morekeywords={}, 
	language=C,                 % the language of the code
	numbers=left,                    % where to put the line-numbers; possible: (none, left, right)
	numbersep=5pt,                   
	numberstyle=\tiny\color{cdgray!50}, 
	rulecolor=\color{cddarkblue}, 
	tabsize=2,
	frameround=tttt
}

\usepackage{qtree}
\usepackage[edges]{forest}

\begin{document}	
	\title{Algorithmen und Datenstrukturen}
	\subtitle{Übung 7: Bäume}
	\author{Eric Kunze}
	\email{eric.kunze@mailbox.tu-dresden.de}
	\city{TU Dresden}
%	\institute{Lehrstuhl für Grundlagen der Programmierung}
	\titlegraphic{\includegraphics[width=2cm]{../TUD-white.pdf}}
	\date{11.12.2020}

	\maketitle


%%%%%%%%%%%%%%%%%%%%%%%%%%%%%%%%%%%%%%%%%%%%%%%%%%%%%%%%%%%%%%%%%%%%%%%%%%%%%

\begin{frame}[fragile, t] \frametitle{Datenstrukturen}
	\begin{lstlisting}[style=notebook]
	typedef struct element *list;
	struct element { int value; list next; };
	\end{lstlisting}
	\pause
	\begin{lstlisting}[style=notebook]
	typedef struct node *tree;
	struct node { int key; tree left, right; };

	\end{lstlisting}
\end{frame}

\begin{frame}[fragile] \frametitle{Teil (a) --- Erstellen eines neuen Knotens}
	\textbf{Verknüpfen zweier Bäume mit einem neuen Wurzelnoten}
	\pause
	\begin{itemize}
		\item lege neuen Wurzelknoten an
		\item verknüpfe Wurzelknoten mit bestehenden Bäumen
		\item fülle neue Wurzel mit Inhalt
	\end{itemize}
	\pause
	\begin{lstlisting}[style=notebook]
	tree createNode(int n, tree l, tree r) {
			tree t   = malloc(sizeof(struct node));
			t->left  = l;
			t->right = r;
			t->key   = n;
			return t;
	}
	\end{lstlisting}
\end{frame}

\begin{frame}[fragile] \frametitle{Teil (a) --- Erstellen eines Baumes}
	\textbf{Beispielbaum:}
	\begin{center}
		\begin{forest}
			for tree={ grow=south, circle, draw, minimum size=3ex, inner sep=1pt, s sep=10mm }
			[4 	[5] [2 [0] [,no edge, draw=none]] ]
		\end{forest}
	\end{center}
	\pause
	\begin{lstlisting}[style=notebook]
	tree bsp = 
		createNode(4,
			createNode(5, NULL, NULL),
			createNode(2,
				createNode(0, NULL, NULL),
				NULL));
	\end{lstlisting}
\end{frame}


\begin{frame}[fragile] \frametitle{Teil (b) --- links Einfügen}
	\begin{tabularx}{\linewidth}{m{2cm} m{2cm} m{1.5cm} m{3cm}}
		\textbf{Beispiel:}
		&
		\begin{forest}
			for tree={ grow=south, circle, draw, minimum size=3ex, inner sep=1pt, s sep=10mm }
			[4 	[5] [2 [0] [,no edge, draw=none]] ]
		\end{forest}
		&
		$\overset{\texttt{insertl(\&bsp)}}{\longrightarrow}$
		&
		\begin{forest}
			for tree={ grow=south, circle, draw, minimum size=3ex, inner sep=1pt, s sep=3mm }
			[4 	[5 [18] [,no edge, draw=none]] [2 [0] [,no edge, draw=none]] ]
		\end{forest}
	\end{tabularx}
	\pause
	\begin{itemize}
		\item solange kein Blatt erreicht: füge in den linken Teilbaum ein
		\item am Blatt: ein neuen Knoten einfügen (\texttt{createNode})
	\end{itemize}
	\pause
	\begin{lstlisting}[style=notebook]
	void insertl(tree *tp, int n) {
			if (*tp != NULL)
					insertl(&((*tp)->left), n);
			else
					*tp = createNode(n, NULL, NULL);
	}
	\end{lstlisting}
\end{frame}

\begin{frame}[fragile] \frametitle{Teil (c) --- Produkt der Blätter}
	\textbf{Beispiele:}
	
	\begin{center}
		\begin{tabular}{ccc}
			bsp & s & t \\ \\
			\begin{forest}
				for tree={ grow=south, circle, draw, minimum size=3ex, inner sep=1pt, s sep=3mm }
				[4 	[5 [18] [,no edge, draw=none]] [2 [0] [,no edge, draw=none]] ]
			\end{forest}
			&
			\begin{forest}
				for tree={ grow=south, circle, draw, minimum size=3ex, inner sep=1pt, s sep=3mm }
				[2 	[3 [2] [4]] [1] ]
			\end{forest}
			&
			\begin{forest}
				for tree={ grow=south, circle, draw, minimum size=3ex, inner sep=1pt, s sep=3mm }
				[2 	[2] [3] ]
			\end{forest} \\ \pause
			\footnotesize \texttt{leafprod(bsp) = 0} &
			\footnotesize \texttt{leafprod(s) = 8} &
			\footnotesize \texttt{leafprod(t) = 6}
		\end{tabular}
	\end{center}
\end{frame}
\begin{frame}[fragile] \frametitle{Teil (c) --- Produkt der Blätter}
	\begin{itemize}
		\item leerer Baum: \texttt{leafprod = 1}
		\item Rekursion: berechne \texttt{leafprod} in linkem und rechtem Teilbaum
	\end{itemize}
	\pause
	\begin{lstlisting}[style=notebook]
	int leafprod(tree t){
			if (t == NULL)
					return 1;
			if (t->left == NULL && t->right == NULL)
					return t->key;
			return leafprod(t->left) * leafprod(t->right);
	}
	\end{lstlisting}
\end{frame}

\begin{frame} \frametitle{Teil (d) --- Blätter entfernen}
	\textbf{Beispiel:}
	\begin{center}
		\begin{tabularx}{\linewidth}{m{3cm} m{1.5cm} m{2cm}}
			\begin{forest}
				for tree={ grow=south, circle, draw, minimum size=3ex, inner sep=1pt, s sep=3mm }
				[4 	[5 [18] [,no edge, draw=none]] [2 [0] [,no edge, draw=none]] ]
			\end{forest}
			&
			$\overset{\texttt{defol(\&bsp)}}{\longrightarrow}$
			&
			\begin{forest}
				for tree={ grow=south, circle, draw, minimum size=3ex, inner sep=1pt, s sep=3mm }
				[4 	[5] [2] ]
			\end{forest}
		\end{tabularx}
	\end{center}
	\pause
	\begin{itemize}
		\item Blatt: Knoten entfernen und Vorgänger auf \texttt{NULL} setzen
		\item Rekursion: Lösche in linkem und rechtem Teilbaum
	\end{itemize}
\end{frame}
\begin{frame}[fragile] \frametitle{Teil (d) --- Blätter entfernen}
	
	\begin{lstlisting}[style=notebook]
	void defol(tree *tp){
		if ( tp == NULL) return ;
		if (*tp == NULL) return ;
		
		if ((*tp)-> left == NULL && (*tp)->right == NULL){
				free(*tp);
				*tp = NULL;
		} else {
				defol( &((*tp)->left)  );
				defol( &((*tp)->right) );
		}
	}
	\end{lstlisting}
\end{frame}

\begin{frame}[fragile] \frametitle{Teil (e) --- Baum $\leadsto$ Liste}
	Man übernehme nur gerade Elemente!
	
	\textbf{Beispiel:}
	\begin{center}
		\begin{tabularx}{\linewidth}{m{2cm} m{2cm} m{2cm}}
			\begin{forest}
				for tree={ grow=south, circle, draw, minimum size=3ex, inner sep=1pt, s sep=3mm }
				[2 	[3 [2] [4]] [1] ]
			\end{forest}
			&
			$\overset{\texttt{treeToList(s)}}{\longrightarrow}$
			&
			\begin{ttfamily}
				[2, 4, 2]
			\end{ttfamily}
		\end{tabularx}
	\end{center}
	\pause
	\textbf{Inorder -- Durchlauf:}
	\begin{itemize}
		\item traversiere durch linken Teilbaum
		\item Wurzel gerade? $\to$ anhängen an Liste mit \texttt{append}
		\item traversiere durch rechten Teilbaum
	\end{itemize}
\end{frame}
\begin{frame}[fragile] \frametitle{Teil (e) --- Baum $\leadsto$ Liste}
	\begin{lstlisting}[style=notebook]
	void append(list *lp, int n)
	\end{lstlisting}
	\pause
	\begin{lstlisting}[style=notebook]
	void treeToList_rec(tree t, list *lp){
			if (t == NULL) return ;
			treeToList_rec(t->left, lp);
			if (t->key % 2 == 0)
					append(lp, t->key);
			treeToList_rec(t->right, lp);
	}	
	\end{lstlisting}
	\pause
	\begin{lstlisting}[style=notebook]
	list treeToList(tree t){
			list l = NULL;
			treeToList_rec(t,&l);
			return l;
	}
	\end{lstlisting}
\end{frame}

\begin{frame}[fragile] \frametitle{Teil (f) --- zähle Vorkommen}
	
	\textbf{Beispiel:}
	\begin{center}
		\begin{tabularx}{\linewidth}{m{1cm} m{2cm} m{0.2cm} m{1.7cm} m{1cm} m{2cm}}
			\texttt{baz} $\Bigg($
			&
			\begin{forest}
				for tree={ grow=south, circle, draw, minimum size=3ex, inner sep=1pt, s sep=3mm }
				[2 	[3 [2] [4]] [1] ]
			\end{forest}
			&
			\texttt{,}
			&
			\begin{forest}
				for tree={ grow=south, circle, draw, minimum size=3ex, inner sep=1pt, s sep=3mm }
				[2 	[2] [3] ]
			\end{forest}
			&
			$\Bigg) \quad =$			
			&
			\begin{forest}
				for tree={ grow=south, circle, draw, minimum size=3ex, inner sep=1pt, s sep=3mm }
				[2 	[1 [2] [0]] [0] ]
			\end{forest}
		\end{tabularx}
	\end{center}
\end{frame}
\begin{frame}[fragile] \frametitle{Teil (f) --- zähle Vorkommen}
	
	\begin{lstlisting}[style=notebook]
	int count(tree t, int k) { 
			/* zaehlt die anzahl der vorkommen von k in t */
			if (t == NULL) return 0;
			return (t->key == k) + count(t->left, k) + count(t->right, k);
	}
	\end{lstlisting}
	\pause
	\begin{lstlisting}[style=notebook]
	tree baz(tree s, tree t) {
			if (s == NULL) return NULL;
			tree result = malloc(sizeof(struct node));
			result->key   = count(t, s->key);
			result->left  = baz(s->left, t) ;
			result->right = baz(s->right, t);
			return result;
	}
	\end{lstlisting}
\end{frame}
%%%%%%%%%%%%%%%%%%%%%%%%%%%%%%%%%%%%%%%%%%%%%%%%%%%%%%%%%%%%%%%%%%%%%%%%%%%%%%%%%%%


%%%%%%%%%%%%%%%%%%%%%%%%%%%%%%%%%%%%%%%%%%%%%%%%%%%%%%%%%%%%%%%%%%%%%%%%%%%%%%%%%%%





%%%%%%%%%%%%%%%%%%%%%%%%%%%%%%%%%%%%%%%%%%%%%%%%%%%%%%%%%%%%%%%%%%%%%%%%%%%%%%%%%%%

\end{document}