\documentclass{beamer}
\usepackage{../tut-slides}
\usepackage{../mathoperatorsAuD}

\usepackage{lmodern}
\usepackage{amsmath,amssymb}
\usepackage{wasysym}
\usepackage{stmaryrd}
\usepackage{enumerate}
%\usepackage[inline]{enumitem} 		%customize label
%\newcommand{\labelitemi}{\raisebox{1pt}{\scalebox{.9}{$\blacktriangleright$}}}
%\newcommand{\labelitemii}{$\vartriangleright$}
%\newcommand{\labelitemiii}{--}
\setbeamertemplate{itemize item}{\raisebox{1pt}{\scalebox{.9}{$\blacktriangleright$}}}
\setbeamertemplate{itemize subitem}{$\vartriangleright$}

\usepackage{booktabs}
\usepackage{tabularx}
\usepackage{tabu}
\newcommand*\head{\rowfont{\bfseries}}
\newcommand*{\tw}{\rowfont{\ttfamily}}
\renewcommand{\tabularxcolumn}[1]{>{\hspace{0pt}}m{#1}}
\usepackage{multirow}

\usepackage{cancel}

\usepackage{empheq}
\newcommand*\widefbox[1]{\fbox{\hspace{2em} #1 \hspace{2em}}}

\usepackage{tcolorbox}
\newtcolorbox{mymathbox}[1][]{colback=white, sharp corners, #1}

\usepackage{xcolor}
\usepackage{listings}
\lstset{numbers=left, 
	numberstyle=\tiny, 
	breaklines=true,
	backgroundcolor=\color{cdgray!20},
	numbersep=5pt,
	language=C,
	tabsize=2,
	basicstyle=\footnotesize\ttfamily,
	showstringspaces=false} 

\usepackage{MnSymbol}

\newcommand{\col}[1]{\textcolor{cdpurple}{#1}}
\newcolumntype{R}[1]{>{\centering\arraybackslash}p{#1}}
\usepackage{tabularx}
\renewcommand{\tabularxcolumn}[1]{m{#1}}

\usepackage{qtree}
\usepackage[edges]{forest}

\usepackage{csquotes}

\begin{document}	
	\title{Algorithmen und Datenstrukturen}
	\subtitle{Übung 9: Suchen \& Ersetzen}
	\author{Eric Kunze}
	\email{eric.kunze@tu-dresden.de}
	\city{TU Dresden}
%	\institute{Lehrstuhl für Grundlagen der Programmierung}
	\titlegraphic{\includegraphics[width=2cm]{../TUD-white.pdf}}
	\date{\formatdate{1}{12}{2021}}

	\maketitle


%%%%%%%%%%%%%%%%%%%%%%%%%%%%%%%%%%%%%%%%%%%%%%%%%%%%%%%%%%%%%%%%%%%%%%%%%%%%%

\section{KMP-Algorithmus \\ \textit{Aufgabe 1}} 

\begin{frame} \frametitle{KMP-Algorithmus}
	\begin{itemize}
		\item Mustersuche in (großen) Texten
		\item Ziel: Verschiebung des Musters um mehr als eine Position bei Nichtübereinstimmung.
		\item Methode: Ermittlung einer Verschiebetabelle {\large $\texttt{Tab[]}$} in \textbf{Phase 1}
		\item Bedeutung des Eintrags {\large $\texttt{Tab[i]=j}$}: \\
		\textit{Bei Nichtübereinstimmung an Stelle $i$ wird Position $j$ des Musters an aktueller Vergleichsstelle angelegt.}
		\item Suchprozess in \textbf{Phase 2}
	\end{itemize}

	\textbf{j-algo:} \url{http://j-algo.binaervarianz.de/}
\end{frame}

\begin{frame} \frametitle{KMP-Algorithmus}
	\small
	Suche das Muster \texttt{aaabaaaa} im Text \texttt{aaabaaabaaacaaabaaaa}.
	\begin{center}
		\footnotesize
		\begin{tabular}{l|cccccccc}
			Position &  0 &  1 &  2 &  3 &  4 &  5 &  6 &  7 \\ \hline
			Pattern  &  a &  a &  a &  b &  a &  a &  a &  a \\ \hline
			Tabelle  & -1 & -1 & -1 &  2 & -1 & -1 & -1 &  3 \\
		\end{tabular}
	\end{center}
	
	\rule{\textwidth}{0.4pt}
	
	\renewcommand*{\arraystretch}{.7}
	\setlength{\tabcolsep}{1pt}
	
	Erster Versuch:
	\begin{center}
		\begin{tabular}{ccccccccccccccccccccc}
			a & a & a & b & a & a & a & \textbf{b} & a & a & a & c & a & a & a & b & a & a & a & a \\
			a & a & a & b & a & a & a & \textbf{a}
		\end{tabular}
	\end{center}
	Tabelleneintrag an Position $7$ ist $3$, d.h. $\texttt{Tab[7]=3}$ --- Lege Position $3$ des Musters an aktueller Vergleichsposition an:
	\begin{center}
		\begin{tabular}{ccccccccccccccccccccc}
			a & a & a & b & a & a & a & b & a & a & a & \textbf{c} & a & a & a & b & a & a & a & a \\
			&   &   &   & a & a & a & b & a & a & a & \textbf{a}
		\end{tabular}
	\end{center}
	Gleicher Prozess noch einmal: Missmatch an Position 7 des Musters --- verschiebe Muster auf Position 3.
\end{frame}

\begin{frame} \frametitle{KMP-Algorithmus (Fortsetzung)}
	\small
	Suche das Muster \texttt{aaabaaaa} im Text \texttt{aaabaaabaaacaaabaaaa}.
	\begin{center}
		\footnotesize
		\begin{tabular}{l|cccccccc}
			Position &  0 &  1 &  2 &  3 &  4 &  5 &  6 &  7 \\ \hline
			Pattern  &  a &  a &  a &  b &  a &  a &  a &  a \\ \hline
			Tabelle  & -1 & -1 & -1 &  2 & -1 & -1 & -1 &  3 \\
		\end{tabular}
	\end{center}

	\rule{\textwidth}{0.4pt}
	
	\renewcommand*{\arraystretch}{.7}
	\setlength{\tabcolsep}{1pt}
	
	Wir legen das Muster also wieder an Position $3$ an:
	\begin{center}
		\begin{tabular}{ccccccccccccccccccccc}
			a & a & a & b & a & a & a & b & a & a & a & \textbf{c} & a & a & a & b & a & a & a & a \\
			&   &   &   &  &  &  &  & a & a & a & \textbf{b} & a & a & a & a
		\end{tabular}
	\end{center}
	Wegen $\texttt{Tab[3]=2}$, lege Muster an Position $2$ an:
	\begin{center}
		\begin{tabular}{ccccccccccccccccccccc}
			a & a & a & b & a & a & a & b & a & a & a & \textbf{c} & a & a & a & b & a & a & a & a \\
			&   &   &   &  &  &  &  &  & a & a & \textbf{a} & b & a & a & a & a
		\end{tabular}
	\end{center}
	Wegen $\texttt{Tab[2]=-1}$, lege Muster an Position $-1$ an:
	\begin{center}
		\begin{tabular}{ccccccccccccccccccccc}
			a & a & a & b & a & a & a & b & a & a & a & c & a & a & a & b & a & a & a & a \\
			&   &   &   &   &   &   &   &   &   &   &   & a & a & a & b & a & a & a & a &
		\end{tabular}
	$\smiley$
	\end{center}
\end{frame}


\begin{frame} \frametitle{KMP-Algorithmus --- Die Zyklenmethode}
	Zwei Phasen:
	\pause
	\begin{itemize}
		\item \textbf{1. Phase:} Markieren der längsten Teilwörter im Pattern, die mit einem Präfix übereinstimmen
		\begin{itemize}
			\item ein Zyklus beginnt an einer Patternposition $\texttt{i}$ falls $\texttt{i} \neq \texttt{0}$ und $\texttt{Pat[0] = Pat[i]}$
			\item ein Zyklus endet an der kleisten Patternposition $\texttt{i+m}$, sodass $\texttt{Pat[m+1]} \neq \texttt{Pat[i+m+1]}$
		\end{itemize}
		\pause
		\item \textbf{2. Phase:} Bestimmung der Tabelleneinträge
		\begin{itemize}
			\item $\texttt{Tab[0] = -1}$
			\item Tabelleneinträge nach einem Zyklus: \\
			\textit{Länge des längsten dort endenden Zyklus}
			\item Tabelleneinträgen in einem Zyklus: \\
			\textit{Tabelleneintrag der derzeitigen Position im längsten laufenden Zyklus}
			\item verbleibende Einträge: \texttt{0}
		\end{itemize}
	\end{itemize}
\end{frame}

\begin{frame} \frametitle{KMP-Algorithmus --- Die Zwei-Finger-Methode}
	Die Methode beruht auf der Gleichung
	\begin{equation*}
		\begin{aligned}
		\small
		\texttt{Tab[i]}
		= \max \menge{-1} \cup \menge{m \; \middle| \;
			\setlength{\arraycolsep}{2pt}
			 \begin{array}{rcl}
				0 \le &m& \le i-1 \\
				b_0 \dots b_{m-i} &=& b_{i-m} \dots b_{i-1} \\
				b_m &\neq& b_j \\
			\end{array} }
		\end{aligned} \tag{$\star$} \label{eq: kmp}
	\end{equation*}
	Daraus ergibt sich nach Initialisierung von $\texttt{Tab[0]} = -1$ für jeden folgenden Eintrag $\texttt{Tab[i]}$ folgendes Verfahren:
	\begin{itemize}
		\item \textit{linker Finger}: wähle $m < i$ in absteigender Reihenfolge (also $i-1$, $i-2$, $\dots$), sodass $\texttt{Pat[i]} \neq \texttt{Pat[m]}$
		\item \textit{Parallelverschiebung beider Finger bis zum linken Rand}: wenn $\texttt{Pat[0\dots m-1]} = \texttt{Pat[i-m\dots i-1]}$, dann fülle $\texttt{Tab[i]} = m$.
		\item wenn keine passende Position $m$ gefunden werden kann, dann fülle $\texttt{Tab[i]} = -1$.
	\end{itemize}
\end{frame}


\begin{frame} \frametitle{Aufgabe 1 \hfill \only<2->{\alert{Lösung}}}
	\textbf{Teil (a)} \hspace{3em}
	Pattern: {\large \texttt{aabaaacaab}} \\[1em]
	
	\begin{center}
		\renewcommand*{\arraystretch}{1.3} \setlength{\tabcolsep}{8pt}
		\begin{tabular}{l|cccccccccc}
			Position &  0 &  1 &  2 &  3 &  4 &  5 &  6 &  7 &  8 &  9 \\ \hline
			Pattern  &  a &  a &  b &  a &  a &  a &  c &  a &  a &  b \\ \hline
			Tabelle  & \only<2->{-1 & -1 &  1 & -1 & -1 &  2 &  2 & -1 & -1 &  1} \\
		\end{tabular}
	\end{center}
\end{frame}


\begin{frame} \frametitle{Aufgabe 1 --- Teil (b) \hfill \only<2->{\alert{Lösung}}}
	\textbf{Teil (b)} % \\[1em]
	
	\vspace{-1.5\baselineskip}
	\begin{center}
		\renewcommand*{\arraystretch}{1.3} \setlength{\tabcolsep}{16pt}
		\begin{tabular}{l|cccccc}
			Position &  0 &  1 &  2 &  3 &  4 &  5 \\ \hline
			Pattern  &  c &  b &  \visible<2->{\textit{c}} &  \visible<2->{\textit{c}} &  \visible<2->{\textit{b}} &  a \\ \hline
			Tabelle  & -1 &  0 & -1 &  1 &  0 &  2 \\
		\end{tabular}
	\end{center}

	\footnotesize
	\begin{itemize}[<2->]
		\item $\texttt{Pat[0\dots 1] = Pat[3\dots 4]}$ wegen $\texttt{Tab[5] = 2}$ (Zyklenmethode), d.h. $\texttt{Pat[3] = Pat[0] = c}$ und $\texttt{Pat[4] = Pat[1] = b}$
		\item wegen $\texttt{Tab[3] = 1}$ ist $\texttt{Pat[2] = \texttt{Pat[0]} = c}$ (Zyklenmethode) 
		\item \textbf{oder:} wegen $\texttt{Tab[3] = 1}$ ist $\texttt{Pat[1]} \neq \texttt{Pat[3]}$ und $\texttt{Pat[2] = Pat[0] = c}$ (Parallelverschiebung in der Zwei-Finger-Methode bzw. Gleichung \eqref{eq: kmp})
	\end{itemize}
\end{frame}

%%%%%%%%%%%%%%%%%%%%%%%%%%%%%%%%%%%%%%%%%%%%%%%%%%%%%%%%%%%%%%%%%%%%%%%%%%%%%%%%%%%

\section{Levenshtein-Distanz \\ \textit{Aufgabe 2}}

\begin{frame} \frametitle{Levenshtein-Distanz}
	\textbf{Kosten} zur Überführung eines Wortes $w = w_1 \dots w_n$ in ein Wort $v = v_1 \dots v_k$ ; schreibe $d(w_1 \dots w_j, v_1 \dots v_i) = d(j,i)$.
	\pause
	\begin{align*}
		d(0,i) &= i \\
		d(j,0) &= j \\
		d(j,i) &= \min \menge{ d(j,i-1) + 1, d(j-1,i) + 1 , d(j-1, i-1) + \delta_{j,i}}
	\end{align*}
	für alle $1 \le j \le n$ und alle $1 \le i \le k$ wobei
	\begin{equation*}
		\delta_{j,i} = \begin{cases}
		1 & \text{wenn } w_j \neq v_i \\
		0 & \text{sonst}
		\end{cases} 
	\end{equation*}
	\pause
	\textbf{Anschaulich:} 
	Überlagerung durch Pattern
	$\to$ Pfeile zeigen ''Ursprung`` des Minimums an
	\begin{equation*}
		w_j \neq v_i : \quad \begin{array}{|c|c|}
			\hline +1 & +1 \\ \hline +1 & ? \\ \hline
		\end{array}
		\qquad \qquad
		w_j = v_i : \quad 
		\begin{array}{|c|c|}
		\hline +0 & +1 \\ \hline +1 & ? \\ \hline
		\end{array}
	\end{equation*}
\end{frame}

\begin{frame} \frametitle{Aufgabe 2}
	Gegeben seien die Wörter $w = \texttt{espen}$ und $v = \texttt{beispiele}$.
	\begin{itemize}
		\item[(a)] Berechnen Sie die Levenshtein-Distanz $d(w,v)$. Geben Sie dazu die Berechnungsmatrix an. Tragen Sie alle Zelleneinträge zusammen mit den dazugehörigen Pfeilen ein.
		\item[(b)] Geben Sie die Levenshtein-Distanz $d(\texttt{espe},\texttt{beispiel})$ an. Beachten Sie, dass \texttt{espe} und \texttt{beispiel} Präfixe von \texttt{espen} bzw. \texttt{beispiele} sind.
		\item[(c)] Geben Sie zwei Alignments zwischen \texttt{espen} und \texttt{beispiele} an, die zu den minimalen Kosten führen. Dabei sollen die Alignments die jeweils angewendeten Editieroperation enthalten.
		\item[(d)] Wieviele Alignments enthält die in Aufgabe (a) angegebene Berechnungsmatrix?
	\end{itemize}
\end{frame}

\begin{frame}[t] \frametitle{Aufgabe 2 \hfill \only<2->{\alert{Lösung}}}
	\textbf{Teil (a)} \hspace{1cm} \only<2->{$d(\text{espen}, \text{beispiele}) = 5$}
	
	\begin{center}
	
	\renewcommand*{\arraystretch}{.7}
	\setlength{\tabcolsep}{3pt}
	\begin{tabu}{>{\bfseries}c|ccccccccccccccccccc}
		\rowfont{\bfseries} $d(j,i)$ &       &       & b     &       & e     &       & i     &       & s     &       & p     &       & i     &       & e     &       & l     &       & e \\ \hline  \\
		& \visible<2->{ 0     & $\rightarrow$ & 1     & $\rightarrow$ & 2     & $\rightarrow$ & 3     & $\rightarrow$ & 4     & $\rightarrow$ & 5     & $\rightarrow$ & 6     & $\rightarrow$ & 7     & $\rightarrow$ & 8     & $\rightarrow$ & 9} \\
		& \visible<2->{       & $\searrow$ &       & $\searrow$ &       &       &       &       &       &       &       &       &       & $\searrow$ &       &       &       & $\searrow$ &  } \\
		e     & \visible<2->{ 1     &       & 1     &       & 1     & $\rightarrow$ & 2     & $\rightarrow$ & 3     & $\rightarrow$ & 4     & $\rightarrow$ & 5     & $\rightarrow$ & 6     & $\rightarrow$ & 7     & $\rightarrow$ & 8 } \\
		& \visible<2->{ $\downarrow$ & $\searrow$ & $\downarrow$ & $\searrow$ & $\downarrow$ & $\searrow$ &       & $\searrow$ &       &       &       &       &       &       &       &       &       &       &  } \\
		s     & \visible<2->{ 2     &       & 2     &       & 2     &       & 2     &       & 2     & $\rightarrow$ & 3     & $\rightarrow$ & 4     & $\rightarrow$ & 5     & $\rightarrow$ & 6     & $\rightarrow$ & 7 } \\
		& \visible<2->{ $\downarrow$ & $\searrow$ & $\downarrow$ & $\searrow$ & $\downarrow$ & $\searrow$ & $\downarrow$ & $\searrow$ & $\downarrow$ & $\searrow$ &       &       &       &       &       &       &       &       &  } \\
		p     & \visible<2->{ 3     &       & 3     &       & 3     &       & 3     &       & 3     &       & 2     & $\rightarrow$ & 3     & $\rightarrow$ & 4     & $\rightarrow$ & 5     & $\rightarrow$ & 6 } \\
		& \visible<2->{ $\downarrow$ & $\searrow$ & $\downarrow$ & $\searrow$ &       & $\searrow$ & $\downarrow$ & $\searrow$ & $\downarrow$ &       & $\downarrow$ & $\searrow$ &       & $\searrow$ &       &       &       & $\searrow$ &  } \\
		e     & \visible<2->{ 4     &       & 4     &       & 3     & $\rightarrow$ & 4     &       & 4     &       & 3     &       & 3     &       & 3     & $\rightarrow$ & 4     & $\rightarrow$ & 5 } \\
		& \visible<2->{ $\downarrow$ & $\searrow$ & $\downarrow$ &       & $\downarrow$ & $\searrow$ &       & $\searrow$ & $\downarrow$ &       & $\downarrow$ & $\searrow$ & $\downarrow$ & $\searrow$ & $\downarrow$ & $\searrow$ &       & $\searrow$ &  } \\
		n     & \visible<2->{ 5     &       & 5     &       & 4     &       & 4     & $\rightarrow$ & 5     &       & 4     &       & 4     &       & 4     &       & 4     & $\rightarrow$ & 5 } \\
	\end{tabu}

	\end{center}

	\only<3->{\textbf{Teil (b)} \hspace{1cm} $d(\text{espe}, \text{beispiel}) = 4$}
\end{frame}

\begin{frame}[t] \frametitle{Aufgabe 2 \hfill \only<2->{\alert{Lösung}}}
	
	\textbf{Teil (c)} \hspace{0.7cm} 		
	Alignments mit minimaler Levenshtein-Distanz:
	
	\visible<2->{%
	\begin{center}
		\small
		\begin{tabular}{ccccccccc}
			$\ast$ & e & $\ast$ & s & p & $\ast$ & e & $\ast$ & n \\
			$|$ & $|$ & $|$ & $|$ & $|$ & $|$ & $|$ & $|$ \\
			b   &  e  &  i  & s & p & i & e & l & e \\
			\textit{i} &   & \textit{i} &   &   & \textit{i}  &  & \textit{i} & \textit{s} \\
		\end{tabular}
	
		\begin{tabular}{ccccccccc}
			$\ast$ & e & $\ast$ & s & p & $\ast$ & e & n & $\ast$ \\
			$|$ & $|$ & $|$ & $|$ & $|$ & $|$ & $|$ & $|$ \\
			b   &  e  &  i  & s & p & i & e & l & e \\
			\textit{i} &   & \textit{i} &   &   & \textit{i}  &  & \textit{s} & \textit{i} \\
		\end{tabular}
	\end{center} %
	}

	\textbf{Teil (d)} \hspace{0.7cm} \visible<2->{2 Alignments = 2 Backtraces}
\end{frame}


\section{Weitere Aufgaben aus der Aufgabensammlung \\ \textit{mit Lösungen}}

\begin{frame} \frametitle{Aufgabe 7.1.13 (AGS)}
	
	\begin{itemize}
		\item[(a)] Bestimmen Sie die mit Hilfe des KMP-Algorithmus berechnete Verschiebetabelle für das Pattern \texttt{abbabbaa}.
		\item[(b)] Mit Hilfe des KMP-Algorithmus ist unten stehende Verschiebetabelle berechnet worden. Die mit einem \enquote{?} markierten Einträge sind unbekannt. Vervollständigen Sie das aus den Symbolen \texttt{a}, \texttt{b} und \texttt{c} bestehende Pattern. 
	\end{itemize}
		
	\begin{center}
		\renewcommand*{\arraystretch}{1.1} \setlength{\tabcolsep}{16pt}
		\begin{tabular}{l|cccccc}
			Position &  0 &  1 &  2 &  3 &  4 &  5 \\ \hline
			Pattern  &  \textit{b} &   &   &  &  &  \textit{c} \\ \hline
			Tabelle  & -1 &  ? & ? &  0 &  ? &  3 \\
		\end{tabular}
	\end{center}
\end{frame}

\begin{frame}[t] \frametitle{Aufgabe 7.1.13 (AGS) \hfill \only<2,3>{\alert{Lösung}}}
	\textbf{Teil (a)} \hspace{3em}
	Pattern: {\large \texttt{abbabbaa}} \\[1em]
	
	\begin{center}
		\renewcommand*{\arraystretch}{1.1} \setlength{\tabcolsep}{10pt}
		\begin{tabular}{l|cccccccc}
			Position &  0 &  1 &  2 &  3 &  4 &  5 &  6 &  7  \\ \hline
			Pattern  &  a &  b &  b &  a &  b &  b &  a &  a  \\ \hline
			Tabelle  & \only<2->{ -1 &  0 &  0 & -1 &  0 &  0 & -1 &  4 } \\
		\end{tabular}
	\end{center}
	
	\pause
	
	\textbf{Teil (b)}
	\vspace{-1.5\baselineskip}
	\begin{center}
		\renewcommand*{\arraystretch}{1.1} \setlength{\tabcolsep}{16pt}
		\begin{tabular}{l|cccccc}
			Position &  0 &  1 &  2 &  3 &  4 &  5 \\ \hline
			Pattern  &  \textit{b} &  \onslide+<3->{a} &  \onslide+<3->{b} &  \onslide+<3->{a} &  \onslide+<3->{b} &  \textit{c} \\ \hline
			Tabelle  & -1 &  ? & ? &  0 &  ? &  3 \\
		\end{tabular}
	\end{center}
	
	\pause
	
	\footnotesize
	\begin{itemize}[<3->]
		\item $\texttt{Pat[0 ... 2] = Pat[2 ... 4]}$ wegen $\texttt{Tab[5] = 3}$ (Zyklenmethode), d.h. $\texttt{Pat[2] = Pat[0] = Pat[4] = b}$
		\item wegen $\texttt{Tab[3] = 0}$ ist $\texttt{Pat[3]} \neq  \texttt{Pat[0] = b}$ und wegen $\texttt{Tab[5] = 3}$ ist $\texttt{Pat[3]} \neq \texttt{Pat[5] = c}$ (Zwei-Finger-Methode bzw. Gleichung \eqref{eq: kmp}) \\
		$\Rightarrow \texttt{Pat[3] = Pat[1] = a}$
	\end{itemize}
\end{frame}


\begin{frame} \frametitle{Aufgabe 7.2.1 (AGS)}
	Gegeben seien die Wörter $w = \texttt{Dinstas}$ und $v = \texttt{Distanz}$.
	\begin{itemize}
		\item[(a)] Berechnen Sie die Levenshtein-Distanz $d(w,v)$ zwischen $w$ und $v$. Geben Sie die Berechnungsmatrix vollständig an.
		\item[(b)] Geben Sie alle Alignments mit minimaler Levenshtein-Distanz zwischen $w$ und $v$ an.
	\end{itemize}
\end{frame}

\begin{frame}[t] \frametitle{Aufgabe 7.2.1 (AGS) \hfill \only<2->{\alert{Lösung}}}
		\centering
		\renewcommand*{\arraystretch}{.7}
		\setlength{\tabcolsep}{3pt}
			\begin{tabular}{l|ccccccccccccccc}
				$d(j,i)$ &       &       & \textbf{D} &       & \textbf{i} &       & \textbf{s} &       & \textbf{t} &       & \textbf{a} &       & \textbf{n} &       & \textbf{z} \\ \hline \\
				& \visible<2->{ 0     & $\rightarrow$ & 1     & $\rightarrow$ & 2     & $\rightarrow$ & 3     & $\rightarrow$ & 4     & $\rightarrow$ & 5     & $\rightarrow$ & 6     & $\rightarrow$ & 7 } \\
				& \visible<2->{$\downarrow$ & \alert<3->{$\searrow$} &       &       &       &       &       &       &       &       &       &       &       &       &  } \\
				\textbf{D}     & \visible<2->{1     &       & 0     & $\rightarrow$ & 1     & $\rightarrow$ & 2     & $\rightarrow$ & 3     & $\rightarrow$ & 4     & $\rightarrow$ & 5     & $\rightarrow$ & 6 } \\
				& \visible<2->{$\downarrow$ &       & $\downarrow$ & \alert<3->{$\searrow$} &       &       &       &       &       &       &       &       &       &       &  } \\
				\textbf{i}     & \visible<2->{2     &       & 1     &       & 0     & $\rightarrow$ & 1     & $\rightarrow$ & 2     & $\rightarrow$ & 3     & $\rightarrow$ & 4     & $\rightarrow$ & 5 } \\
				& \visible<2->{$\downarrow$ &       & $\downarrow$ &       & \alert<3->{$\downarrow$} & $\searrow$ &       & $\searrow$ &       & $\searrow$ &       & $\searrow$ &       &       &  } \\
				\textbf{n}     & \visible<2->{3     &       & 2     &       & 1     &       & 1     & $\rightarrow$ & 2     & $\rightarrow$ & 3     &       & 3     & $\rightarrow$ & 4 } \\
				& \visible<2->{$\downarrow$ &       & $\downarrow$ &       & $\downarrow$ & \alert<3->{$\searrow$} &       & $\searrow$ &       & $\searrow$ &       & $\searrow$ & $\downarrow$ & $\searrow$ &  } \\
				\textbf{s}     & \visible<2->{4     &       & 3     &       & 2     &       & 1     & $\rightarrow$ & 2     & $\rightarrow$ & 3     & $\rightarrow$ & 4     &       & 4 } \\
				& \visible<2->{$\downarrow$ &       & $\downarrow$ &       & $\downarrow$ &       & $\downarrow$ & \alert<3->{$\searrow$} &       &       &       &       &       &       &  } \\
				\textbf{t}     & \visible<2->{5     &       & 4     &       & 3     &       & 2     &       & 1     & $\rightarrow$ & 2     & $\rightarrow$ & 3     & $\rightarrow$ & 4 } \\
				& \visible<2->{$\downarrow$ &       & $\downarrow$ &       & $\downarrow$ &       & $\downarrow$ &       & $\downarrow$ & \alert<3->{$\searrow$} &       &       &       &       &  } \\
				\textbf{a}     & \visible<2->{6     &       & 5     &       & 4     &       & 3     &       & 2     &       & 1     & \alert<3-3>{$\rightarrow$} & 2     & $\rightarrow$ & 3 } \\
				& \visible<2->{$\downarrow$ &       & $\downarrow$ &       & $\downarrow$ & $\searrow$ & $\downarrow$ &       & $\downarrow$ &       & $\downarrow$ & \alert<4->{$\searrow$} &       & \alert<3-3>{$\searrow$} &  } \\
				\textbf{s}     & \visible<2->{7     &       & 6     &       & 5     &       & 4     &       & 3     &       & 2     &       & 2     & \alert<4->{$\rightarrow$} & 3 } \\
			\end{tabular}
		
		\pause
		\begin{equation*}
			d(\text{Dinstas}, \text{Distanz}) = 3
		\end{equation*}
\end{frame}

\begin{frame}[t] \frametitle{Aufgabe 7.2.1 (AGS) \hfill \only<2->{\alert{Lösung}}}
	Alignments mit minimaler Levenshtein-Distanz:
	
	\centering
	\visible<2->{
	\begin{tabular}{cccccccc}
		D & i & n & s & t & a & $\ast$ & s \\
		$|$ & $|$ & $|$ & $|$ & $|$ & $|$ & $|$ & $|$ \\
		D & i & $\ast$ & s & t & a & n & z \\
		  &   & \textit{d} &   &   &   & \textit{i} & \textit{s} \\
	\end{tabular}

	\vspace{3em}
	
	\begin{tabular}{cccccccc}
		D & i & n & s & t & a & s & $\ast$ \\
		$|$ & $|$ & $|$ & $|$ & $|$ & $|$ & $|$ & $|$ \\
		D & i & $\ast$ & s & t & a & n & z \\
		&   & \textit{d} &   &   &   & \textit{s} & \textit{i} \\
	\end{tabular}
	}
\end{frame}


\begin{frame} \frametitle{Aufgabe 7.2.2 (AGS)}
	\begin{itemize}
		\item[(a)] Berechnen Sie die Levenshtein-Distanz $d(\texttt{bürste}, \texttt{schürze})$. Geben Sie die Berechnungsmatrix vollständig an. Wieviele Backtraces enthält die Berechnungsmatrix?
		\item[(b)] Geben Sie zwei Alignments mit minimaler Levenshtein-Distanz zwischen den Wörtern
		\texttt{bürst} und \texttt{sch} an.
	\end{itemize}
\end{frame}

\begin{frame}[t] \frametitle{Aufgabe 7.2.2 (AGS) --- Teil (a) \hfill \only<2->{\alert{Lösung}}}
	\centering
	\renewcommand*{\arraystretch}{.7}
	\setlength{\tabcolsep}{3pt}
	\begin{tabular}{l|rrrrrrrlrrrlrlr}
		$d(j,i)$ &       &       & \textbf{s} &       & \textbf{c} &       & \textbf{h} &       & \textbf{ü} &       & \textbf{r} &       & \textbf{z} &       & \textbf{e} \\ \hline \\
		& \visible<2->{0     & \alert<3->{$\rightarrow$} & 1     & \alert<3->{$\rightarrow$} & 2     & $\rightarrow$ & 3     & $\rightarrow$ & 4     & $\rightarrow$ & 5     & $\rightarrow$ & 6     & $\rightarrow$ & 7 } \\
		& \visible<2->{$\downarrow$ & \alert<4->{$\searrow$} &       & \alert<4->{$\searrow$} &       & \alert<3->{$\searrow$} &       & $\searrow$ &       & $\searrow$ &       & $\searrow$ &       & $\searrow$ &  } \\
		\textbf{b}     & \visible<2->{1     &       & 1     & \alert<4->{$\rightarrow$} & 2     & \alert<4->{$\rightarrow$} & 3     & $\rightarrow$ & 4     & $\rightarrow$ & 5     & $\rightarrow$ & 6     & $\rightarrow$ & 7 } \\
		& \visible<2->{$\downarrow$ & $\searrow$ & $\downarrow$ & $\searrow$ &       & $\searrow$ &       & \alert<3->{$\searrow$} &       &       &       &       &       &       &  } \\
		\textbf{ü}     & \visible<2->{2     &       & 2     &       & 2     & $\rightarrow$ & 3     &       & 3     & $\rightarrow$ & 4     & $\rightarrow$ & 5     & $\rightarrow$ & 6 } \\
		& \visible<2->{$\downarrow$ & $\searrow$ & $\downarrow$ & $\searrow$ & $\downarrow$ & $\searrow$ &       & $\searrow$ & $\downarrow$ & \alert<3->{$\searrow$} &       &       &       &       &  } \\
		\textbf{r}     & \visible<2->{3     &       & 3     &       & 3     &       & 3     & $\rightarrow$ & 4     &       & 3     & $\rightarrow$ & 4     & $\rightarrow$ & 5 } \\
		& \visible<2->{$\downarrow$ & $\searrow$ &       & $\searrow$ & $\downarrow$ & $\searrow$ & $\downarrow$ & $\searrow$ &       &       & \alert<4->{$\downarrow$} & \alert<3->{$\searrow$} &       & $\searrow$ &  } \\
		\textbf{s}     & \visible<2->{4     &       & 3     & $\rightarrow$ & 4     &       & 4     &       & 4     &       & 4     &       & 4     & $\rightarrow$ & 5 } \\
		& \visible<2->{$\downarrow$ &       & $\downarrow$ & $\searrow$ &       & $\searrow$ & $\downarrow$ & $\searrow$ & $\downarrow$ & $\searrow$ & $\downarrow$ & \alert<4->{$\searrow$} & \alert<3->{$\downarrow$} & $\searrow$ &  } \\
		\textbf{t}     & \visible<2->{5     &       & 4     &       & 4     & $\rightarrow$ & 5     &       & 5     &       & 5     &       & 5     &       & 5 } \\
		& \visible<2->{$\downarrow$ &       & $\downarrow$ & $\searrow$ & $\downarrow$ & $\searrow$ &       & $\searrow$ & $\downarrow$ & $\searrow$ & $\downarrow$ & $\searrow$ & $\downarrow$ & \alert<3->{$\searrow$} &  } \\
		\textbf{e}     & \visible<2->{6     &       & 5     &       & 5     &       & 5     & $\rightarrow$ & 6     &       & 6     & & 6     &  & 5 } \\
	\end{tabular}%

	\pause

	\begin{equation*}
		d(\text{bürste}, \text{schürze}) = 5 \qquad \pause \pause \pause \text{Anzahl der Backtraces } = 3 * 2 = 6
	\end{equation*}
\end{frame}

\begin{frame}[t] \frametitle{Aufgabe 7.2.2 (AGS) --- Teil (b) \hfill \only<2->{\alert{Lösung}}}
	Alignments mit minimaler Levenshtein-Distanz zwischen den Wörtern
	\texttt{bürst} und \texttt{sch}
	
	\visible<2->{
	\centering
	\renewcommand*{\arraystretch}{.7}
	\setlength{\tabcolsep}{3pt}
	\begin{tabular}{l|ccccccc}
		$d(j,i)$ &       &       & \textbf{s} &       & \textbf{c} &       & \textbf{h}  \\ \hline \\
		& 0     & $\rightarrow$ & 1     & $\rightarrow$ & 2     & $\rightarrow$ & 3  \\
		& \alert<2->{$\downarrow$} & \alert<2->{$\searrow$} &       & $\searrow$ &       & $\searrow$ &  \\
		\textbf{b}     & 1     &       & 1     & $\rightarrow$ & 2     & $\rightarrow$ & 3  \\
		& \alert<2->{$\downarrow$} & \alert<2->{$\searrow$} & \alert<2->{$\downarrow$} & \alert<2->{$\searrow$} &       & $\searrow$ &   \\
		\textbf{ü}     & 2     &       & 2     &       & 2     & $\rightarrow$ & 3  \\
		& \alert<2->{$\downarrow$} & \alert<2->{$\searrow$} & \alert<2->{$\downarrow$} & \alert<2->{$\searrow$} & \alert<2->{$\downarrow$} & \alert<2->{$\searrow$} & \\
		\textbf{r}     & 3     &       & 3     &       & 3     &       & 3 \\
		& $\downarrow$ & \alert<2->{$\searrow$} &       & \alert<2->{$\searrow$} & \alert<2->{$\downarrow$} & \alert<2->{$\searrow$} & \alert<2->{$\downarrow$}  \\
		\textbf{s}     & 4     &       & 3     & \alert<2->{$\rightarrow$} & 4     &       & 4  \\
		& $\downarrow$ &       & $\downarrow$ & \alert<2->{$\searrow$} &       & \alert<2->{$\searrow$} & \alert<2->{$\downarrow$} \\
		\textbf{t}     & 5     &       & 4     &       & 4     & \alert<2->{$\rightarrow$} & 5   \\
	\end{tabular}%
	}
\end{frame}

\begin{frame}[t] \frametitle{Aufgabe 7.2.2 (AGS) --- Teil (b) \hfill \only<2->{\alert{Lösung}}}
	Alignments mit minimaler Levenshtein-Distanz zwischen den Wörtern
	\texttt{bürst} und \texttt{sch}
	
	\centering
	\visible<2->{
	\begin{tabular}{ccccc}
		b & ü & r & s & t \\
		$|$ & $|$ & $|$ & $|$ & $|$ \\
		s & c & h & $\ast$ & $\ast$ \\
		\textit{s} & \textit{s} & \textit{s} & \textit{d} & \textit{d}  \\
	\end{tabular}
	\hspace{2em}
	\begin{tabular}{ccccc}
		b & ü & r & s & t \\
		$|$ & $|$ & $|$ & $|$ & $|$ \\
		$\ast$ & $\ast$ & s & c & h \\
		\textit{d} & \textit{d} & \textit{s} & \textit{s} & \textit{s}  \\
	\end{tabular}
	}
\end{frame}

\end{document}