\documentclass{beamer}
\usepackage{../tut-slides}
\usepackage{../mathoperatorsAuD}

\usepackage{amsmath,amssymb}
\usepackage{stmaryrd}
\usepackage{enumerate}
%\usepackage[inline]{enumitem} 		%customize label
%\newcommand{\labelitemi}{\raisebox{1pt}{\scalebox{.9}{$\blacktriangleright$}}}
%\newcommand{\labelitemii}{$\vartriangleright$}
%\newcommand{\labelitemiii}{--}
\setbeamertemplate{itemize item}{\raisebox{1pt}{\scalebox{.9}{$\blacktriangleright$}}}
\setbeamertemplate{itemize subitem}{$\vartriangleright$}

\usepackage{booktabs}
\usepackage{tabularx}
\usepackage{tabu}
\newcommand*\head{\rowfont{\bfseries}}
\newcommand*{\tw}{\rowfont{\ttfamily}}
\renewcommand{\tabularxcolumn}[1]{>{\hspace{0pt}}m{#1}}

\usepackage{cancel}

\usepackage{empheq}
\newcommand*\widefbox[1]{\fbox{\hspace{2em} #1 \hspace{2em}}}

\usepackage{tcolorbox}
\newtcolorbox{mymathbox}[1][]{colback=white, sharp corners, #1}

%%%% EBNF-Terme %%%%
\newcommand{\wdh}[1]{\hat{\{} \ #1 \ \hat{\}}}
\newcommand{\opt}[2]{\hat{(} \ #1 \ \hat{|} \ #2 \ \hat{)}}
\newcommand{\byp}[1]{\hat{[} \ #1 \ \hat{]}}
\newcommand{\rdb}[1]{\hat{(} \ #1 \ \hat{)}}

\newcommand{\sem}[1]{\left\llbracket #1 \right\rrbracket}

\newcommand*{\ttfamilywithbold}{\fontfamily{lmtt}\selectfont}

\usepackage{xcolor}
\usepackage{listings}
\lstset{numbers=left, 
	numberstyle=\tiny, 
	breaklines=true,
	numbersep=5pt,
	language=C,
	tabsize=2,
	basicstyle=\normalsize\ttfamilywithbold} 

\lstdefinestyle{example}{
	basicstyle=\small\ttfamilywithbold,   
	breakatwhitespace=false,         % sets if automatic breaks should only happen at whitespace
	breaklines=true,                 % sets automatic line breaking
	commentstyle=\itshape,    	     % comment style
	escapeinside={\%*}{*)},          % if you want to add LaTeX within your code
	extendedchars=true,              % lets you use non-ASCII characters; for 8-bits encodings only, does not work with UTF-8
	backgroundcolor=\color{cdgray!10},
	keywordstyle=\bfseries,       % keyword style
	morekeywords={}, 
	language=C,                 % the language of the code
	numbers=left,                    % where to put the line-numbers; possible: (none, left, right)
	numbersep=5pt,                   % how far the line-numbers are from the code
	numberstyle=\tiny\color{cdgray!50}, % the style that is used for the line-numbers
	rulecolor=\color{cddarkblue}, 
	tabsize=2,	                   % sets default tabsize to 2 spaces
}
\lstdefinestyle{frame}{
	basicstyle=\normalsize\ttfamilywithbold,   
	breakatwhitespace=false,         % sets if automatic breaks should only happen at whitespace
	breaklines=true,                 % sets automatic line breaking
	commentstyle=\itshape,    	     % comment style
	escapeinside={\%*}{*)},          % if you want to add LaTeX within your code
	extendedchars=true,              % lets you use non-ASCII characters; for 8-bits encodings only, does not work with UTF-8
	frame=single,
	keywordstyle=\bfseries,       % keyword style
	morekeywords={}, 
	language=C,                 % the language of the code
	numbers=left,                    % where to put the line-numbers; possible: (none, left, right)
	numbersep=5pt,                   % how far the line-numbers are from the code
	numberstyle=\tiny\color{cdgray!50}, % the style that is used for the line-numbers
	rulecolor=\color{cddarkblue}, 
	tabsize=2,	                   % sets default tabsize to 2 spaces
}


\begin{document}	
	\title{Algorithmen und Datenstrukturen}
	\subtitle{Übung 4: Programmieren mit $C$}
	\author{Eric Kunze}
	\email{eric.kunze@tu-dresden.de}
	\city{TU Dresden}
%	\institute{Lehrstuhl für Grundlagen der Programmierung}
	\titlegraphic{\includegraphics[width=2cm]{../TUD-white.pdf}}
	\date{\today}

	\maketitle


%%%%%%%%%%%%%%%%%%%%%%%%%%%%%%%%%%%%%%%%%%%%%%%%%%%%%%%%%%%%%%%%%%%%%%%%%%%%%

\begin{frame}[fragile] \frametitle{Programmieren mit C = Crogrammieren}
	\begin{itemize}
		\item Input / Output: \lstinline{#include <stdio.h>}
		\item Variablentypen: z.B. \lstinline{int}, \lstinline{float}, \lstinline{char}
		\item arithmetische Operatoren: \lstinline{+}, \lstinline{-},  \lstinline{*}, \lstinline{/}, \lstinline{%}
		\item Vergleichsoperatoren: \lstinline{==}, \lstinline{<}, \lstinline{<=}, \lstinline{>}, \lstinline{>=}
		\item Logikoperatoren: 
		\begin{itemize}
			\item Negation: \lstinline{!} 
			\item Konjunktion: \lstinline{&&} 
			\item Disjunktion: \lstinline{||}
		\end{itemize}
		\item Arrays: \lstinline{int feld[7]} (Indizierung beginnend bei 0)
	\end{itemize}
\end{frame}

\begin{frame}[fragile] \frametitle{Input \& Output}
	... benötigt \lstinline|#include <stdio.h>|
	\begin{itemize}
		\item Einlesen: \lstinline{scanf("%d", &n);}
		\item Ausgeben: \lstinline{printf("n!=%d\n", factorial);}
	\end{itemize}

	weitere Informationen: \\
	\url{https://www.tutorials.at/c/03-dateneingabe-ausgabe.html}
			
			
\end{frame}

%%%%%%%%%%%%%%%%%%%%%%%%%%%%%%%%%%%%%%%%%%%%%%%%%%%%%%%%%%%%%%%%%%%%%%%%%%%%%%%%%%%

\begin{frame}[fragile] \frametitle{Bedingungen}
	\small
	\begin{itemize}
		\item \fbox{\texttt{if}-\texttt{else}-Statement:} \\
		bedingte Ausführung eines Statements
\begin{lstlisting}[style=example]
if ( BoolExp ) {
	Statement ;
} else  {
	Statement ;
}
\end{lstlisting}
		\item \fbox{\texttt{switch}-Statement:} \\
		Fallunterscheidung mit mehr als zwei Fällen
\begin{lstlisting}[style=example]
switch ( Exp ) {
	case 0: StatementSeq ; break ;
	case 1: StatementSeq ; break ;
	default: StatementSeq ;
}
\end{lstlisting}
	\end{itemize}
\end{frame}

%%%%%%%%%%%%%%%%%%%%%%%%%%%%%%%%%%%%%%%%%%%%%%%%%%%%%%%%%%%%%%%%%%%%%%%%%%%%%%%%%%%

\begin{frame}[fragile] \frametitle{Schleifen}
	\begin{itemize}
		\item \fbox{\texttt{while}-Statement:} wiederholte Ausführung eines Statements (Schleifenrumpf)
\begin{lstlisting}[style=example]
while ( BoolExp ) {
	Statement ;
}
\end{lstlisting}
		\item \fbox{\texttt{do}-\texttt{while}-Statement:} vergleichbar mit While-Statement, aber Schleifenbedingung wird nach Rumpf geprüft
\begin{lstlisting}[style=example]
do {
	Statement ;
} while ( BoolExp )
\end{lstlisting}

	\end{itemize}
\end{frame}

%%%%%%%%%%%%%%%%%%%%%%%%%%%%%%%%%%%%%%%%%%%%%%%%%%%%%%%%%%%%%%%%%%%%%%%%%%%%%%%%%%%

\begin{frame}[fragile] \frametitle{Schleifen}
	\begin{itemize}
		\item \fbox{\texttt{for}-Statement:} vor der Schleife steht Anzahl der Schleifendurchläufe fest
\begin{lstlisting}[style=example]
for ( Assignment ; BoolExp ; Assignment ) {
	Statement ;
}
\end{lstlisting}	
	zum Beispiel:
\begin{lstlisting}[style=example]
for (int i = 1; i <= n; i = i + 1){
	printf("i = %d\n", i);
}
\end{lstlisting}	
	\end{itemize}
\end{frame}

%%%%%%%%%%%%%%%%%%%%%%%%%%%%%%%%%%%%%%%%%%%%%%%%%%%%%%%%%%%%%%%%%%%%%%%%%%%%%%%%%%%

\section{Übungsblatt 4}

\begin{frame}[fragile] \frametitle{Aufgabe 1 --- Teil (a)}
	\textbf{Maximum:} $\displaystyle \max(n,m) = \begin{cases}
		m & \text{wenn } n < m \\
		n & \text{sonst}
	\end{cases}$ \pause
	
	\vspace{\baselineskip}
	
	\begin{lstlisting}[style=frame, basicstyle=\ttfamilywithbold\footnotesize]
#include <stdio.h>
int main() {
	int x, y, m;
	scanf("%d", &x);
	scanf("%d", &y);
	if ( x < y ) {
		m = y;
	} else { // x >= y
		m = x;
	}
	printf("Das Maximum von %d und %d ist %d.\n", x, y, m);
	return 0;
}
	\end{lstlisting}
\end{frame}

\begin{frame}[fragile] \frametitle{Aufgabe 1 --- Teil (b)}
	\textbf{Fakultät:} $\displaystyle n! = n * (n-1) * (n-2) * \dots * 2 * 1$ \pause
	
\begin{lstlisting}[style=frame]
#include <stdio.h>
int main(){
	int n, i, factorial = 1;
	scanf("%d", &n);
	for (i = 1; i <= n; i = i + 1){
		factorial = factorial * i;
	}
	printf("n! = %d\n", factorial);
	return 0;
}
\end{lstlisting}
\end{frame}

\begin{frame}[fragile] \frametitle{Aufgabe 1 --- Teil (c)}
	\pause
	
	\begin{lstlisting}[style=frame]
#include <stdio.h>
int main(){
	int i,j,n;
	scanf("%d", &n);
	for ( i=1; i<=n ; i=i+1 ) {
		for ( j=1; j<=n ; j=j+1 ) {
			printf("%4d", i*j);
		}
		printf("\n");
	}
	return 0;
}
	\end{lstlisting}
\end{frame}


\begin{frame}[fragile] \frametitle{Aufgabe 1 --- Teil (d)}
	\pause
	
	\begin{lstlisting}[style=frame, basicstyle=\ttfamilywithbold\footnotesize]
#include <stdio.h>
int main(){
	for (int i = 2; i <= 1000; i++) {
		int j = 2, prime = 1;
		while (j*j <= i) {
			if (i%j == 0) {
				prime = 0;
				break; // nicht notwendig
			}
			j++;
		}
		if (prime == 1)
			printf("%d ist prim\n", i);
	}
	return 0;
}
	\end{lstlisting}
\end{frame}

\begin{frame}[fragile] \frametitle{Aufgabe 2}
	\begin{lstlisting}[style=frame, showstringspaces=false, basicstyle=\ttfamilywithbold\scriptsize]
#include <stdio.h>
#include <stdlib.h>

int main() {
	unsigned int guess, number, n;
	scanf("%u", &n);
	number = 1 + rand() % n;
	while(1) {
		printf("Zahl raten: ");
		scanf("%u", &guess);
		if (guess == number) {
			printf("Sehr gut!\n");
			return 0;
		} else if (guess > number) {
			printf("Zahl ist kleiner!\n");
		} else {
			printf("Zahl ist groesser!\n");
		}
	}
	return 0;
}
	\end{lstlisting}%
\end{frame}

\begin{frame}[fragile] \frametitle{Zusatzaufgabe 1}
	\footnotesize
	\begin{itemize}
		\item \textbf{A:} \lstinline|matches > 0|
		\item \textbf{B:} \lstinline|turn == 1|
		\item \textbf{C:} 
		\begin{lstlisting}[basicstyle=\ttfamilywithbold\tiny, numbers=none]
z = (matches - 1) % 4;
if (z == 0)
	z = 1
matches = matches - z;
turn = 0;
		\end{lstlisting}
	\item \textbf{D:} 
	\begin{lstlisting}[basicstyle=\ttfamilywithbold\tiny, numbers=none]
printf("Der Computer zog %d Strichhoelzer;\n", z);
printf("somit verbleiben %d Streichhoelzer.\n", matches);
	\end{lstlisting}
	\item \textbf{E:} 
	\begin{lstlisting}[basicstyle=\ttfamilywithbold\tiny, numbers=none]
scanf("%d", &z);
matches = matches - z;
turn = 1;
	\end{lstlisting}
	\item \textbf{F:}
	\begin{lstlisting}[basicstyle=\ttfamilywithbold\tiny, numbers=none]
if (turn == 1)
	printf("Der Computer gewinnt.");
else
	printf("Der Mensch gewinnt.");
	\end{lstlisting}
	\end{itemize}
\end{frame}
\end{document}