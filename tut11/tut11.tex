\documentclass{beamer}
\usepackage{../tut-slides}
\usepackage{../mathoperatorsAuD}

\usepackage{lmodern}
\usepackage{amsmath,amssymb}
\usepackage{wasysym}
\usepackage{stmaryrd}
\usepackage{enumerate}
%\usepackage[inline]{enumitem} 		%customize label
%\newcommand{\labelitemi}{\raisebox{1pt}{\scalebox{.9}{$\blacktriangleright$}}}
%\newcommand{\labelitemii}{$\vartriangleright$}
%\newcommand{\labelitemiii}{--}
\setbeamertemplate{itemize item}{\raisebox{1pt}{\scalebox{.9}{$\blacktriangleright$}}}
\setbeamertemplate{itemize subitem}{$\vartriangleright$}

\usepackage{pgfplots}
\pgfplotsset{compat=1.10}   % in my packages used compat=1.15
\usepgfplotslibrary{fillbetween}
\usepackage{pgf}
\usepackage{tikz}
\usetikzlibrary{patterns,arrows, arrows.meta,calc,decorations.pathmorphing,backgrounds, positioning,fit,automata,shapes,matrix}
\tikzset{onslide/.code args={<#1>#2}{%
		\only<#1| handout:1>{\pgfkeysalso{#2}} 
}}

\usepackage{booktabs}
\usepackage{tabularx}
\usepackage{tabu}
\newcommand*\head{\rowfont{\bfseries}}
\newcommand*{\tw}{\rowfont{\ttfamily}}
\renewcommand{\tabularxcolumn}[1]{>{\hspace{0pt}}m{#1}}
\usepackage{multirow}

\usepackage{cancel}

\usepackage{empheq}
\newcommand*\widefbox[1]{\fbox{\hspace{2em} #1 \hspace{2em}}}

\usepackage{tcolorbox}
\newtcolorbox{mymathbox}[1][]{colback=white, sharp corners, #1}

\usepackage{xcolor}
\usepackage{listings}
\lstset{numbers=left,
	numberstyle=\tiny,
	breaklines=true,
	backgroundcolor=\color{cdgray!20},
	numbersep=5pt,
	language=C,
	tabsize=2,
	basicstyle=\footnotesize\ttfamily,
	showstringspaces=false}

\DeclareMathOperator{\ack}{\mathbf{ack}}
\usepackage{MnSymbol}

\newcommand{\col}[1]{\textcolor{cdpurple}{#1}}
\newcolumntype{R}[1]{>{\centering\arraybackslash}p{#1}}
\usepackage{tabularx}
\renewcommand{\tabularxcolumn}[1]{m{#1}}

\usepackage{qtree}
\usepackage[edges]{forest}
\usepackage{csquotes}

\begin{document}
	\title{Algorithmen und Datenstrukturen}
	\subtitle{Übung 11: Kürzeste Wege}
	\author{Eric Kunze}
	\email{eric.kunze@tu-dresden.de}
	\city{TU Dresden}
%	\institute{Lehrstuhl für Grundlagen der Programmierung}
	\titlegraphic{\includegraphics[width=2cm]{../TUD-white.pdf}}
	\date{\formatdate{5}{1}{2022}}

	\maketitle


%%%%%%%%%%%%%%%%%%%%%%%%%%%%%%%%%%%%%%%%%%%%%%%%%%%%%%%%%%%%%%%%%%%%%%%%%%%%%



%%%%%%%%%%%%%%%%%%%%%%%%%%%%%%%%%%%%%%%%%%%%%%%%%%%%%%%%%%%%%%%%%%%%%%%%%%%%%%%%%%%

\section{Dijkstra-Algorithmus}

\begin{frame}[t] \frametitle{Suchverfahren}
	\small	
	\begin{minipage}[t]{\dimexpr0.5\linewidth-\fboxrule-\fboxsep}
		\textbf{Tiefensuche:}
		\begin{itemize}
			\item gehe in die Tiefe: \\
			\enquote{entdecke erst Kinder, dann Geschwister}
			\item Nachfolger werden \emph{oben} auf den Keller gelegt
			\item nächster Knoten wird \emph{oben} vom Keller genommen
		\end{itemize}
		
		\bigskip 
		
		\textbf{Keller:}
		\begin{center}
			\scalebox{0.5}{%
			\begin{tikzpicture}[scale=0.5]
				\draw[thick] (-1,-1) -- (3,-1);
				\foreach \y in {0,...,5}
				\draw (0,\y) rectangle ++(2,-1);
				\node[anchor=north east] (nachfolger) at (-3,5) {Nachfolger};
				\draw[->, arrows=-Stealth, bend angle=45, bend left]  (nachfolger) to (1,5);
				\node[anchor=south west] (next) at (4,4) {nächster Knoten};
				\draw[->, arrows=-Stealth]  (2,4.5) to (next);
			\end{tikzpicture}}
		\end{center}
	\end{minipage}
	\begin{minipage}[t]{\dimexpr0.5\linewidth-\fboxrule-\fboxsep}
		\textbf{Breitensuche:}
		\begin{itemize}
			\item gehe in die Breite: \\
			\enquote{entdecke erst Geschwister, dann Kinder}
			\item Nachfolger stellen sich \emph{hinten} an
			\item nächster Knoten wird von \emph{vorn} genommen
		\end{itemize}
		
		\bigskip 
		
		\textbf{Warteschlange:}
		\begin{center}
			\scalebox{0.4}{%
			\begin{tikzpicture}
				\foreach \x in {0,...,5}
				\draw[color=defaultcolor] (\x,0) rectangle ++(1,1);
				\node[anchor=south west] (nachfolger) at (9,1) {Nachfolger};
				\draw[->, arrows=-Stealth, bend angle=45, bend right]  (nachfolger.west) to (6.5,1);
				\node[anchor=north] (next) at (-1.5,-1) {nächster Knoten};
				\draw[->, arrows=-Stealth, bend angle=45, bend right]  (0,0.5) to (next.north);
			\end{tikzpicture}}
		\end{center}
	\end{minipage}
\end{frame}

\begin{frame} \frametitle{Verallgemeinerte Graphensuche}
	\small
	\textbf{Beobachtung}: Suche läuft ähnlich ab
	\begin{itemize}
		\item Operation 1: Lesen des nächsten Knotens \hfill \texttt{READ}
		\item Operation 2: Löschen des gewählten Knotens \hfill \texttt{REMOVE} \\ $\phantom{Operation 2: \enskip}$ \textcolor{cdgray!50}{(und seiner Duplikate)} 
		\item Operation 3: Hinzufügen der Nachfolgerknoten \hfill \texttt{INSERT}
		\item Operation 4: Leerheit der Datenstruktur prüfen \hfill \texttt{EMPTY}
	\end{itemize}
	
	\pause
	\begin{center}
		\begin{tabular}{l|ccccc}
			& \texttt{STORAGE} & \texttt{READ} & \texttt{REMOVE} & \texttt{INSERT} & \texttt{EMPTY} \\ \hline
			Tiefensuche & Keller & \texttt{top} & \texttt{pop} & \texttt{push} & \texttt{empty} \\
			Breitensuche & Queue & \texttt{head} & \texttt{dequeue} & \texttt{enqueue} & \texttt{nil} \\
		\end{tabular}
	\end{center}
\end{frame}

\begin{frame} \frametitle{Graphensuche mit Prioritätswarteschlange}
	\small
	weitere Möglichkeit für \texttt{STORAGE}: \textbf{Prioritätswarteschlange}
	\begin{itemize}
		\item \texttt{READ} -- Auswahl des nächsten Knotens mit minimaler Priorität 
		\item \texttt{REMOVE} -- as usual
		\item \texttt{INSERT} -- Nachfolger stellt sich entsprechend seiner Priorität in die Warteschlange \\ (oder Prioritätswert erhält ein Update, wenn er bereits in der Warteschlange steht)
	\end{itemize}

	\textbf{Vorstellung}: \textit{\enquote{geordnete}} Warteschlange
\end{frame}

\begin{frame} \frametitle{Dijkstra-Algorithmus}
	\small
	Graphensuche mit Prioritätswarteschlange:
	\begin{equation*}
		\text{Priorität} = \text{Priorität des Vorgängers} + \text{Kantengewicht}
	\end{equation*}

	\pause
	
	\textbf{Beispiel}:
	\begin{center}
		\scalebox{0.6}{%
			\begin{tikzpicture}[->, 
				>=stealth', 
				initial text=, 
%					node distance=3cm, 
				bend angle=20, 
				semithick,
				elliptic state/.style={draw,ellipse},
				chosen/.style={ultra thick, fill=cdorange!40},
				done/.style={fill=defaultcolor!10}
				]%[node distance=2cm,auto]
				
				\node[elliptic state, accepting, onslide={<3>{chosen}}, onslide={<4->{done}}] (dd) {Dresden}; 
				\node[elliptic state, onslide={<7>{chosen}}, onslide={<8->{done}}] (pd) [above right = 1cm and 1cm of dd] {Potsdam}; 
				\node[elliptic state, onslide={<5>{chosen}}, onslide={<6->{done}}] (le) [right = 2cm of dd] {Leipzig};
				\node[elliptic state, onslide={<9>{chosen}}, onslide={<10->{done}}] (be) [right = 2cm of pd] {Berlin};
				
				\tikzset{every node/.style = {font=\footnotesize, color=cdblue}};
				\path[->]
				(dd) edge node[above left] {160km} (pd) 
				(dd) edge node[above] {100km} (le)
				(pd) edge node[above] {30km} (be)
				(le) edge node[below right] {150km} (be);
		\end{tikzpicture}}
	\end{center}

	Wir notieren Knoten in der Form $(\text{Knoten}, \text{Priorität}, \text{Vorgänger})$.
	
	\scalebox{0.7}{%
	\renewcommand{\arraystretch}{1.4}
	\begin{tabular}{l|l}
		\hline
		gewählter Knoten & Warteschlange \\ \hline\hline
		\onslide<3->{(Dresden, 0 km, --)} & \onslide<4->{\alert<5>{(Leipzig, 0 + 100 km, Dresden)}, (Potsdam, 0 + 160 km, Dresden)} \\ \hline
		\onslide<5->{(Leipzig, 100 km, Dresden)} & \onslide<6->{\alert<7>{(Potsdam, 160 km, Dresden)}, (Berlin, 100 + 150 km, Leipzig)} \\ \hline
		\onslide<7->{(Potsdam, 160 km, Dresden)} & \onslide<8->{\alert<9>{(Berlin, 160 + 30 km, Potsdam)}} \\ \hline
		\onslide<9->{(Berlin, 190 km, Potsdam)} & \onslide<10->{---} \\ \hline
	\end{tabular}}
\end{frame}

\begin{frame} \frametitle{Aufgabe 1}
	\small
	\begin{minipage}{\dimexpr0.5\linewidth-\fboxrule-\fboxsep}
		\begin{center}
			\scalebox{0.7}{%
				\begin{tikzpicture}[->, 
					>=stealth', 
					initial text=, 
					%					node distance=3cm, 
					semithick,
					chosen/.style={ultra thick, fill=cdorange!40},
					done/.style={fill=defaultcolor!10}
					]
					
					\node[state, onslide={<3>{chosen}}, onslide={<4->{done}}] (1) {$1$}; 
					\node[state, onslide={<9>{chosen}}, onslide={<10->{done}}] (2) [above left = 1cm and 2cm of 1] {$2$}; 
					\node[state, onslide={<5>{chosen}}, onslide={<6->{done}}] (3) [above = 2cm of 1] {$3$};
					\node[state, onslide={<7>{chosen}}, onslide={<8->{done}}] (4) [above right = 1cm and 2cm of 1] {$4$};
					\node[state, onslide={<11>{chosen}}, onslide={<12->{done}}] (5) [below left = 1cm and 2cm of 1] {$5$};
					\node[state, onslide={<13>{chosen}}, onslide={<14->{done}}] (6) [below right = 1cm and 2cm of 1] {$6$};
					
					\tikzset{every node/.style = {font=\footnotesize, color=cdblue}};
					\path[->]
					(1) edge node[above left] {$10$} (4) 
					(1) edge node[above left] {$15$} (5)
					(1) edge node[left] {$3$} (3)
					(2) edge node[left] {$4$} (5)
					(3) edge node[above left] {$6$} (2)
					(3) edge node[above right] {$5$} (4)
					(4) edge node[right] {$6$} (6)
					(5) edge node[above] {$2$} (6);
			\end{tikzpicture}}
		\end{center}
	\end{minipage} \pause
	\begin{minipage}{\dimexpr0.5\linewidth-\fboxrule-\fboxsep}
		\scalebox{0.7}{%
			\begin{tabular}{l|l}
				\hline
				gewählter Knoten & Randknotenmenge \\ \hline\hline
				\onslide<3->{$(1, 0, --)$} & \onslide<4->{\alert<5>{$(3, 3, 1)$}, $(4, 10, 1)$, $(5, 15, 1)$} \\ \hline
				\onslide<5->{$(3, 3, 1)$} & \onslide<6->{$(2, 9, 3)$, \alert<7>{$(4, 8, 3)$}, $(5, 15, 1)$} \\ \hline
				\onslide<7->{$(4, 8, 3)$} & \onslide<8->{\alert<9>{$(2, 9, 3)$}, $(5, 15, 1)$, $(6, 14, 4)$} \\ \hline
				\onslide<9->{$(2, 9, 3)$} & \onslide<10->{\alert<11>{$(5, 13, 2)$}, $(6, 14, 4)$} \\ \hline
				\onslide<11->{$(5, 13, 2)$} & \onslide<12->{\alert<13>{$(6, 14, 4)$}} \\ \hline
				\onslide<13->{$(6, 14, 4)$} & \onslide<14->{---} \\ \hline
		\end{tabular}}
	
		\bigskip
		
		\onslide<15->{%
			\textbf{Pfadtabelle}:
			\medskip
			
			\scalebox{0.7}{%
				%			\renewcommand{\arraystretch}{1.4}
				\begin{tabular}{c|c|l}
					\hline
					Zielknoten & Pfadlänge & kürzester Pfad \\ \hline\hline
					$1$ & $0$ & $1$ \\ \hline
					$2$ & $9$ & $1, 3, 2$ \\ \hline
					$3$ & $3$ & $1, 3$ \\ \hline
					$4$ & $8$ & $1, 3, 4$ \\ \hline
					$5$ & $13$ & $1, 3, 2, 5$ \\ \hline
					$6$ & $13$ & $1, 3, 4, 6$ \\ \hline
			\end{tabular}}
		}
	\end{minipage}
\end{frame}

%%%%%%%%%%%%%%%%%%%%%%%%%%%%%%%%%%%%%%%%%%%%%%%%%%%%%%%%%%%%%%%%%%%%%%%%%%%%%%%%%%%

\section{Floyd-Warshall-Algorithmus}

\begin{frame} \frametitle{Floyd-Warshall-Algorithmus}
	\small
	\begin{itemize}
		\item gewichteter Graph $G = (V,E,c)$ mit Weglängen $c$ und ohne Schlingen
		\item \textbf{Ziel}: kürzeste Wege von beliebigem Startknoten zu beliebigem Zielknoten
		\item oBdA: $V = \menge{1, \dots, n}$
		\pause
		\item $P_{u,v} =$ Menge aller Wege von $u$ nach $v$
		\item $D_G (u,v) = \begin{cases}
		\min\menge{c_p : p \in P_{u,v}} & \text{wenn } P_{u,v} \neq \emptyset \\
		\infty & \text{sonst}
		\end{cases}$
		\pause
		\item $P^{(k)}_{u,v} =$ Menge aller Wege von $u$ nach $v$, deren innere Knoten in $\menge{1, \dots, k}$ liegen
		\item $D_G^{(k)} (u,v) = \begin{cases}
		\min\menge{c_p : p \in P_{u,v}^{(k)}} & \text{wenn } P_{u,v}^{(k)} \neq \emptyset \\
		\infty & \text{sonst}
		\end{cases}$
		\pause
		\item Es gilt $P^{(n)}_{u,v} = P_{u,v}$ und somit $D_G^{(n)} = D_G$.
	\end{itemize}
\end{frame}

\begin{frame} \frametitle{Floyd-Warshall-Algorithmus}
	\begin{itemize}
		\item modifizierte Adjazenzmatrix $mA_G = \min\menge{A_G, \mathbf{0}_n} = \begin{cases}
		c(u,v) & \text{wenn } u \neq v, (u,v) \in E \\
		0 & \text{wenn } u = v \\
		\infty & \text{sonst}
		\end{cases}$
		\item \textbf{Initialisierung}: $D_G^{(0)} = mA_G$
		\item \textbf{Rekursion}:
		\begin{equation*}
			\boxed{D_G^{(k+1)}(u,v) = \min\menge{D_G^{(k)}(u,v) , D_G^{(k)}(u,k+1) + D_G^{(k)}(k+1,v)}}
		\end{equation*}
	\end{itemize}
\end{frame}
\begin{frame} \frametitle{Aufgabe 2 \onslide<2->{--- Teil (a)}}
	\begin{minipage}{\dimexpr0.42\linewidth-\fboxrule-\fboxsep}
		\begin{center}
			\scalebox{0.9}{%
				\begin{tikzpicture}[->, 
					>=stealth', 
					initial text=, 
					%					node distance=3cm, 
					semithick,
					chosen/.style={ultra thick, fill=cdorange!40},
					done/.style={fill=defaultcolor!10}
					]
					
					\node[state] (4) {$4$}; 
					\node[state] (1) [above left = 1cm and 1cm of 4] {$1$};
					\node[state] (3) [above = 2cm of 4] {$3$};
					\node[state] (6) [above right = 1cm and 1cm of 4] {$6$};
					\node[state] (2) [below left = 1cm and 1cm of 4] {$2$};
					\node[state] (5) [below = 2cm of 4] {$5$};
					\node[state] (7) [below right = 1cm and 1cm of 4] {$7$};
					
					\tikzset{every node/.style = {font=\footnotesize, color=cdblue}};
					
					% Kanten
					\path[->] 
						(1) edge node[above left]{$3$} (3)
						(2) edge node[left] {$8$} (1)
						(2) edge node[above left]{$2$} (4)
						(4) edge node[left] {$8$} (3)
						(4) edge node[above right]{$4$} (1)
						(4) edge node[above left]{$3$} (6)
						(4) edge node[above right]{$6$} (7)
						(5) edge node[below left]{$4$} (2)
						(5) edge node[left] {$7$} (4)
						(5) edge node[below right]{$15$} (7)
						(6) edge node[above right]{$3$} (3)
						(6) edge node[right] {$2$} (7);
			\end{tikzpicture}}
		\end{center}
	\end{minipage}
	\pause
	\begin{minipage}{\dimexpr0.58\linewidth-\fboxrule-\fboxsep}
		\small
		$mA_G = \begin{pmatrix}
		0      & \infty & 3      & \infty & \infty & \infty & \infty \\
		8      & 0      & \infty & 2      & \infty & \infty & \infty \\
		\infty & \infty & 0      & \infty & \infty & \infty & \infty \\
		4      & \infty & 8      & 0      & \infty & 3      & 6      \\
		\infty & 4      & \infty & 7      & 0      & \infty & 15     \\
		\infty & \infty & 3      & \infty & \infty & 0      & 2      \\
		\infty & \infty & \infty & \infty & \infty & \infty & 0
		\end{pmatrix}$
	\end{minipage}
\end{frame}

\begin{frame} \frametitle{Aufgabe 2 --- Teil (b)}
	\begin{equation*}
		D_G^{(2)} = \begin{pmatrix}
		0      & \infty & 3      & \infty & \infty & \infty & \infty \\
		8      & 0      & \textcolor{cdpurple}{11}     & 2      & \infty & \infty & \infty \\
		\infty & \infty & 0      & \infty & \infty & \infty & \infty \\
		4      & \infty & \textcolor{cdpurple}{7}     & 0      & \infty & 3      & 6      \\
		\textcolor{cdgreen}{12}     & 4      & \textcolor{cdgreen}{15}     & \textcolor{cdgreen}{6}      & 0      & \infty & 15     \\
		\infty & \infty & 3      & \infty & \infty & 0      & 2      \\
		\infty & \infty & \infty & \infty & \infty & \infty & 0
		\end{pmatrix}
	\end{equation*}
	d.h. es ändern sich folgende Einträge:
	\begin{equation*}
		\underbrace{\textcolor{cdpurple}{(4,3,7), (2,3,11)}}_{\text{aus } D_G^{(1)}}, \underbrace{\textcolor{cdgreen}{(5,3,15), (5,1,12), (5,4,6)}}_{\text{aus } D_G^{(2)}}
	\end{equation*}
\end{frame}

\begin{frame} \frametitle{Aufgabe 2 --- Teil (c)}
	Für $k \in \menge{4,6}$, d.h. durch Zulassen der Knoten $4$ und $6$ als innere Knoten eines Weges, erreichen wir eine Verbesserung. Dagegen sind die Knoten $3$ und $7$ \textit{Senken}, d.h. es bringt nichts, diese zu besuchen, weil wir nicht wieder weg kommen. Ebenso bringt uns der Knoten $5$ als \textit{Quelle} keine Verbesserung, weil wir diesen gar nicht erreichen können.
	Somit gilt also
	\begin{equation*}
		D_G^{(2)} = D_G^{(3)} \qquad D_G^{(4)} = D_G^{(5)} \qquad D_G^{(6)} = D_G^{(7)}
	\end{equation*}
	und wir müssen lediglich $D_G^{(4)}$ sowie $D_G^{(6)}$ explizit berechnen.
\end{frame}

\begin{frame} \frametitle{Aufgabe 2 --- Teil (d)}
	\small
	\begin{equation*}
	\begin{aligned}
		D_G^{(4)} &= \begin{pmatrix}
		0      & \infty & 3      & \infty & \infty & \infty & \infty \\
		\alert{6}      & 0      & \alert{9}     & 2      & \infty & \alert{5} & \alert{8} \\
		\infty & \infty & 0      & \infty & \infty & \infty & \infty \\
		4      & \infty & 11     & 0      & \infty & 3      & 6      \\
		\alert{10}     & 4      & \alert{13}     & 6      & 0      & \alert{9} & \alert{12}     \\
		\infty & \infty & 3      & \infty & \infty & 0      & 2      \\
		\infty & \infty & \infty & \infty & \infty & \infty & 0
		\end{pmatrix}
		= D_G^{(5)} \\
		D_G^{(6)} &= \begin{pmatrix}
		0          & \infty & 3          & \infty & \infty & \infty & \infty \\
		6          & 0      & \alert{8}  & 2      & \infty & 5      & \alert{7} \\
		\infty     & \infty & 0          & \infty & \infty & \infty & \infty \\
		4          & \infty & \alert{6}  & 0      & \infty & 3      & \alert{5}     \\
		10         & 4      & \alert{12} & 6      & 0      & 9      & \alert{11}     \\
		\infty     & \infty & 3          & \infty & \infty & 0      & 2      \\
		\infty     & \infty & \infty     & \infty & \infty & \infty & 0
		\end{pmatrix}
		= D_G^{(7)} = D_G
	\end{aligned}
	\end{equation*}
\end{frame}
\end{document}